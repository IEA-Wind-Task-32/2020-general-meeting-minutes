\subsection{Panel discussion: \enquote{Wind lidar - the next generation}}

We started with presentations from all panelists with their view of how the wind lidar and wind energy community should be teaching and training the next generation of wind lidar users. 53 participants joined us for this session.

\subsubsection{Presentations}

Clym Stock-Williams, TNO

\begin{itemize}
	\item Scientist in industry must know the limitations and assumptions of their equipment, especially for lidar systems
	\item Regular training courses on lidar related technology are needed targeted at industry professionals
	\item Data scientists and statisticians are largely missing from wind energy industry
\end{itemize}

Sandrine Aubrun, ECN

\begin{itemize}
	\item Do not set meteorology and engineering sciences as opposites or exclusives in education programs - both subjects are important but are taught in different courses
	\item Better transfer of knowledge from the research community to the industrial end-users
\end{itemize}

Marijn Floris van Dooren, ForWind

\begin{itemize}
	\item Should lidar theory and wind energy application be an integral part of uni programs?
	\item Do we need a lidar course for a non-academic audience?
	\item Existing European/international networks such as the European Wind Energy Master and the ITN project LIKE push the expertise and exploitation of lidar and enhance diversity in the field.
\end{itemize}

Sarah Barber, OST

\begin{itemize}
	\item Improving diversity in wind energy science
	\item Why do we need to improve diversity? The workforce does not represent our population's diversity which results from inequality
	\item Why should I care? Diverse teams are more productive
	\item What can I do? Increase awareness, get clued up, observe and report discriminations
\end{itemize}

\subsubsection{Discussions}

\emph{Many of the following questions and chat were taken verbatim from the video chat window. There have been some edits for spelling and clarity.}

Sarah to Sandrine: How should we set up those programs?

\begin{itemize}
	\item Sandrine: We have to actively facilitate transfer of knowledge. This could be a task for the LIKE project.
	\item Peter Rosenbusch: I am very grateful for the collaboration between academia and industry. A technology workshop is ongoing. We are offering webinars at Leosphere, and are happy to do more of those.
	\item Marijn: There are two industry workshops planned in LIKE where the goal is to transfer knowledge between the groups. One project might not be enough!
\end{itemize}

Andy to Clym: are we reaching enough people, or is the lidar community too small?

\begin{itemize}
	\item Clym: it is great to hear that industry if offering courses. But the question is, if those courses also teach others' technology. Wind field reconstruction is a very important topic as well that needs to be taught. And each device needs to be treated differently.
\end{itemize}

Question from the chat: Why is knowledge of wind energy not so open and accessible in online platforms like Coursera or EDX, compared to solar energy? I know this is something irrelevant to current discussion but I would love to hear from current members?

\begin{itemize}
	\item From Sarah Barber (via chat): Hi \ldots, this is a really good question and very relevant to the topic, in my opinion. We at OST are
 actually involved in trying to solve this problem by building a wind energy collaboration platform including data and workflow sharing. I can tell you more about it in private if you are interested. 
	 \item Zachary - I wonder if a collaboration with IEA Wind Task 43 Digitalisation might be interesting for this? Data sharing and collaboration is a part of this task.
	\item From Sandrine: I think this is a very good idea. This should be the objective of the EAWE (European Academy for Wind Energy) or other academic institutions. Such a course could be the goal to be constructed.
	\item Sarah: there are not many wind energy courses. So it is not surprising that there is nothing online so far. The question might also be, if we need more of those basic courses
	\item Marijn: I agree, there are not many programs. In Oldenburg there are good courses, but this is not part of a specialization. The European Wind Energy Master program (EWEM is a collaboration between TU Delft, DTU, NTNU, and the University of Oldenburg) is a successful example of how knowledge can be combined within Europe. But more combined or shared programs would be good.
	\item Clym: There might be a difference between solar and wind because solar is more for domestic use. A wind energy master would be extremely useful for a university. In Delft there is also a course that has to be paid for. In my experience the students from master courses have a broad knowledge. A basic bachelor knowledge and a master in wind is often not enough knowledge to go into research. The specialization should take place on PhD level.
	\item Andy: for lidar we need to come up with material that sums up the state of the art.
	\item David: Master students are often looking for topics but cannot find some. The Task 32 could offer to be a platform for advertising master thesis topics in the newsletter
	\item Zachary responded: Like the one I recently posted on Linkedin!
\end{itemize}

Question from the chat: This seems to be a matter of managing interfaces. Research sometimes needs to be separate from industry to encourage innovation without certain limits, and then it needs to exchange at a certain point to be used practically. How can we use IEA Task 32 to guide/frame the interface?

\begin{itemize}
	\item Marijn: indeed this interface is missing. Often practicalities make it very hard to test things or implement ideas
	\item Andy: We need playgrounds where industry and academia can meet safely on a legal basis.
	\item Sarah: we need a way for industry and academia to work together with common data. I think it is possible to have a platform or set up where this is possible.
	\item Andy: we are starting to ask questions about digitalisation of lidar. We will be spinning this up over the next year.
\end{itemize}

Question from the chat: LiDAR technology for wind originally came from the atmospheric boundary layer research community. Today, the wind energy science community is somehow \enquote{separated} (maybe not the right term) from the ABL research world (with some exceptions like the collaboration with DWD for \href{https://www.windfors.de/en/projects/wipaff/}{WIPAFF}). Do you think it would make sense to reconnect with the ABL met folks, for instance through projects like the EU \href{https://www.cost.eu/actions/CA18235/\#tabs\%7CName:overview}{PROBE COST Action}? They have wind lidars too, but also use lidars for other things, and have other interesting tech like radiometers for instance. How much overlap do you think with those groups?

\begin{itemize}
	\item Sandrine: I think this is exactly the idea which I had for the educational program which is split between earth sciences and engineering. A lot of people in wind energy a lot of people come from physics or earth science - so the link exists already but is probably not used enough or established.
	\item Andy: Often we are most comfortable to talk to people who are doing something similar. The Task 32 OA is trying to get involved with PROBE but this may take some time. We encourage everyone to get involved with other activities where they see links and share knowledge from, or with, Task 32.
\end{itemize}

Andy: This brings us back to diversity. Sarah brought up the point, that if you don't have the whole society represented, you do not get what you need. Do you think we are wearing a white western hat?

\begin{itemize}
	\item Sarah: Well, you are wearing a white male, western european heterosexual hat. And that is unconscious. Everybody should be conscious about it.
	\item Andy: as white male engineers - what might I be doing that stops different people from engaging?
	\item Sarah: Starts with language. A lot of people refer to engineers as he. You might write a job description which focuses more on male behaviours. I had a job description myself recently with only male applications. And so the topic might be not written in an interesting way to appeal to female people.
\end{itemize}

Andy to all participants: what was your experience with trying to get a diverse applicant pool into your projects?

\begin{itemize}
	\item Marijn: All universities tried to take care of diversity, and the \href{https://www.msca-like.eu/}{ITN LIKE} project is relatively diverse.
	\item Sandrine: in FLOAWER we tried to increase the percentage of accepted women compared to how many applied. The key element of selection was not the gender but the knowledge. We still managed to improve the percentage. I got feedback from positive discremination by being too many women in my group. Sometimes we are being used as representatives. For my career this was a positive aspect.
	\item Ines agrees: It is important to start early. For example at the University of Stuttgart girls from school are introduced to science at an early age through the \href{https://www.uni-stuttgart.de/studium/orientierung/try-science/}{Try Science} program.
	\item Clym: how can we help as a lidar community? My feeling for outreach work is that lidar is a very physical subject. Everyone experiences the wind, the magic of lidar ist that you can feel it. And this is inspiring. There is an african society which is also trying to foster diversity - so we should really try to reach out of our own borders.
\end{itemize}

An engineer (via chat): Hiring practices need to be less intuition based - see e.g., \enquote{Thinking, Fast and Slow} Chapter 21, by Daniel Kahnemann\cite{kahneman2011thinking}. What role could IEA Task 32 really play in encouraging this?

\begin{itemize}
	\item Sarah Barber: The first step is even getting people to accept that under-representation \emph{is} a problem. Many people do \emph{not} believe that something has to be done, because encouraging under-represented groups is seen as \enquote{positive discrimination} or discriminating \emph{against} the white male.
\end{itemize}

Andy: What would the panel members like to provide as a \enquote{take away}?

\begin{itemize}
	\item Marijn: We as a lidar community should make sure that we provide a safe environment for everybody
	\item Sarah: Increasing diversity is something we can do every day - let's get started.
\end{itemize}

\begin{taskactions}
\textbf{Task 32 action}: we will:

\begin{itemize}
	\item Encourage all of our members to get in contact if they would like to use our LinkedIn feed or newsletter to advertise open positions.
	\item Explore the need for structured further education that can be supported by the Task.
	\item look at our activities again from the perspective of diversity and inclusion to make sure that we encourage and enable everyone to take part in the Task. If any of our members have comments, questions, or critique, please \href{mailto:ieawind.task32@ifb.uni-stuttgart.de}{contact the Operating Agents}.
\end{itemize}
\end{taskactions}