\subsection{Reporting and Next steps}

\begin{table}[!h]
    \centering
    % set up banded rows for the agenda and add lines to the columns
    \arrayrulecolor{Task32Blue2!15}
    \rowcolors{2}{Task32Blue2!5}{white}
    \begin{tabular}{@{}|p{0.125\columnwidth}|p{0.85\columnwidth}|@{}}
    \rowcolor{Task32Blue2} \textbf{Time} & \textbf{Activity} \\    
    16:00 & Reporting and next steps \\
    16:55 & Break
    \end{tabular}
    \label{tab:day4-results-agenda}
\end{table}

\emph{43 participants joined us to hear about the outcomes from the working groups}

The groups reported their progress. Each group had 1 slide and 3 minutes to present their work. Following are notes from each group including the summary slide that they prepared. The slides have been reproduced without editing. Each group appointed a Rapporteur to present the slide.

\hypertarget{forecasting}{%
\subsubsection{Forecasting}\label{forecasting}}

\emph{Rapporteur: Ines Würth}

This is a topic with lots of open questions, but there's not much public
research in this area at the moment. Task 32 remains a great place to
share ideas. More information about the discussion in this group is
available in the
\href{https://docs.google.com/document/d/1Yq2JyWJAEZVAJE9te4FJ-57OpXdxNRHWhks88l2YJcI/edit?usp=sharing}{working
group's notes}.

\begin{longtable}[]{@{}l@{}}
\toprule
\textbf{Task 32 action} : we'll store those open questions in a public
space and make them available for others to build on.\tabularnewline
\midrule
\endhead
\bottomrule
\end{longtable}

\hypertarget{wind-lidar-for-wind-energy-applications-in-cold-climate}{%
\subsubsection{Wind lidar for wind energy applications in cold
climate}\label{wind-lidar-for-wind-energy-applications-in-cold-climate}}

\emph{Rapporteur: Nicolas Jolin}

See \protect\hyperlink{_89scdv7d3yn6}{the presentation from Day 2} for
more information about this working group. Studies are ongoing. Please
get in contact with Nicolas Jolin if you are interested.

\begin{longtable}[]{@{}l@{}}
\toprule
\textbf{Task 32 action} : Task 32 will continue to support this working
group.\tabularnewline
\midrule
\endhead
\bottomrule
\end{longtable}

\hypertarget{a-world-without-cups}{%
\subsubsection{A world without cups}\label{a-world-without-cups}}

\emph{Rapporteur: Mads Sorensen, Remi Gandoin}

\begin{itemize}
\tightlist
\item
  These were more philosophical discussions. Important when looking into
  the future
\item
  An entire industry needs to be changed!

\item
  the laser technology and the great progress it lead to in physics;
\item
  going beyond the "lidars don't give me TI" approach by
  questioning/better understanding what these TI values are used for
  (typically: IEC turbulence classes evaluation). In effect, in the IEC
  framework, the input flow models that are used for the load
  simulations are really "toy models" of the atmosphere (steady state
  10min, power law shear, Kaimal neutral form spectra w/ pre-defined
  length scales). We (I) are missing practical examples of situations
  where cups lead to better siting/WTG choice than LiDARs.
\end{itemize}

\textbf{Task 32 action} : Task 32 will continue to explore this
question. We may identify a work case that is not well-served by cup
anemometers and the current approach to wind characterisation, and
investigate how to leverage wind lidar instead.

\subsubsection{Colalboration on wind lidar hardware and software}

\emph{Rapporteur: Francisco Costa}

There's a lot of work going on in this area. The major challenge is to
coordinate activities and tools, and enable them to work together.

\includegraphics{RackMultipart20201125-4-1c19q8c_html_d59cfda985618214.png}

\textbf{Task 32 action} : we'll update the Task 32 glossary to include a
generic lidar design structure, and align it with the open lidar
modules. This glossary can be used to define classes for lidars, like
\emph{optics.telescope.aperture} which could help with defining inputs
for simulations, etc. See
\href{https://github.com/IEA-Wind-Task-32/wind-lidar-glossary}{\url{https://github.com/IEA-Wind-Task-32/wind-lidar-glossary}}
for ongoing efforts.

\subsubsection{Turbulence intensity derived from wind lidar}

\emph{Rapporteur: Reesa Dexter}

This has been a recurrent theme through the General Meeting

\includegraphics{RackMultipart20201125-4-1c19q8c_html_7605375138aabec9.png}

A workshop bringing industry and academia together would be a good next step.

\textbf{Task 32 action} : we'll include the suggested next steps in our
roadmap and start to plan events for 2021 and beyond.

\subsubsection{Wind lidar in complex terrain}

\emph{Rapporteur: Alexander Stoekl}

\includegraphics{RackMultipart20201125-4-1c19q8c_html_c11e362697bc4a26.png}


\textbf{Task 32 action} : Task 32 will continue to support this working
group.

\subsubsection{Floating lidar}

\emph{Rapporteur: Julia Gottschall, IWES Fraunhofer}

The majority of actions here are taking place through the IEC, and it is
not feasible to have parallel activities through Task 32 as many
stakeholders are already taking part in the IEC process. The suggestion
is a third workshop in the second half of 2021 to align.

Please contact Julia or the Operating Agents if you are interested in
collaborating.

\textbf{Task 32 action}: Task 32 will organise an alignment workshop in
the second half of 2021

\subsubsection{Nacelle lidar in complex terrain}
\emph{Rapporteur: Jacob Burrows}


\begin{itemize}
\item
  Comment from a participant: the 2.5 D is not a function of decay but a
  function of the turbine itself. Depends on the turbine size.
\item
  The need is there to perform power curve verification in complex
  terrain from industry perspective
\item
  There could be another workshop on this topic
\end{itemize}


\textbf{Task 32 action}: Task 32 will combine the outcome from this
group with the outcomes from the group looking at power performance
verification using measurements in the induction zone. We'll also
combine this with previous plans to run a round-robin on this theme.
We'll propose a path forward in 2021.

\subsubsection{Power performance verification in the induction zone}

\emph{Rapporteur: Sebastian Streitz}

There is a need for an alternative proxy for freestream wind speed is in
common with nacelle lidar → need for more studies, could also be topic
for a short focused meeting

\textbf{Task 32 action}: Task 32 will combine the outcome from this
group with the outcomes from the group looking at nacelle mounted lidar
in complex terrain. We'll also combine this with previous plans to run a
round-robin on this theme. We'll propose a path forward in
2021.

\subsubsection{Lidar assisted control}

\begin{itemize}

\item
  The reference turbine that will be supplied is being worked on. A
\end{itemize}


\textbf{Task 32 action} : Task 32 will...

\begin{enumerate}
\item
  Continue to work on a open repository of lidar-assisted control
  simulations
\item
  Address the cost of the lidar by a white paper: show that it has come
  down, improve lidar cost modeling
\item
  Organize a white paper to connect turbine OEM's needs to lidar
  manufacturers, e.g. Improve availability, maintenance friendly, more
  adjustable
\item
  Collaborate more with other IEA Wind tasks (IEA Wind Task 37 & the new "wind farm flow control" Task)
\end{enumerate}

The meeting closed at 17:00 CEST on 22 October.